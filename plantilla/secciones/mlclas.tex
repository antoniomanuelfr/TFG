\chapter{Clasificación multi-etiqueta}
El proceso para obtener un modelo para este tipo de clasificación es similar al que se ha visto en secciones anteriores, donde la mayor parte de los algoritmos necesitas una fase de entrenamiento inicial y dependiendo de las etiquetas asignadas a cada muestra, se ajustan los parámetros del modelo.
\linebreak
Antes de pasar a explicar los distintos enfoques con los que se pueden abordar este tipo de problemas, se va a explicar una serie de conceptos que van a ser usados en las secciones siguientes:
\begin{itemize}
	\item \textbf{labelset}: Conjunto de etiquetas a los que se asigna cada muestra. En el caso particular de este problema, sabemos que estamos ante una clasificación ordinal cuyos valores son $\{1,2,3,4,5,6,7\}$ y con un total de 5 etiquetas. El labelset que se va a usar queda definido como: $\{1,2,3,4,5,6,7\}^5$.
\end{itemize}
\section{Enfoques}
A la hora de clasificar datos con múltiples etiquetas, se ha optado por dos enfoques: \textbf{transformar los datos} y \textbf{adaptar los métodos/algoritmos existentes}. \linebreak
El primero de los enfoques se centra en aplicar técnicas de transformación sobre el conjunto de datos con el que se esta trabajando para obtener así uno o varios conjuntos de datos a los que se les puede aplicar clasificación multi-clase o binaria. \linebreak
El segundo, se centra en modificar los algoritmos de clasificación para que sean capaces de manejar estos conjuntos de datos, produciendo así más de una salida.
\subsection{Transformación del conjunto de datos}
Este tipo de clasificación es una tarea más compleja que la clasificación tradicional, y una de las primeras propuestas para resolver este problema es el de transformar los datos para obtener así uno (o varios) problemas más simples.
Los enfoques que se han propuesto son:
\begin{enumerate}
	\item \textbf{Desplegar las muestras multi-etiqueta} Este enfoque descompone cada instancia en tantas instancias como etiquetas contiene, clonando los atributos asociados a cada muestra. La salida será un problema multi-clase, conteniendo más muestras que el conjunto de datos original.
	\item \textbf{Usar el \textit{labelset} como identificador de clase}: Partimos de la idea de usar cada combinación de \textit{labelset} que se encuentre en el conjunto de datos. De esta manera, se obtiene un conjunto de datos multi-clase con el mismo número de instancias y con tantas clases como combinaciones de labelset se encuentren el conjunto de datos original. Este enfoque también se conoce como \textbf{\textit{Label PowerSet}}.
	\item \textbf{Aplicar técnicas de binarización}: Al igual que para el problema multi-clase, se pueden adaptar estas técnicas. El método más común se llama \textbf{Binary Relevance}.
\end{enumerate}
\subsection{Agrupación de algoritmos}
El uso de conjunto de clasificadores junto con una estrategia para juntar sus predicciones se ha probado ser muy efectivos en problemas clásicos.
\section{Métricas usadas}
\section{}
\subsection{Primer enfoque}
En este primer caso, se va a usar uno de los enfoques más simples. Se va a entrenar un clasificador ordinal por cada variable.
Dado que puede existir relaciones entre etiquetas, la princpal desventaja de este enfoque es que no se tiene en cuenta esta relación, perdiendo así información que puede ser importante.\\
\linebreak
En el problema que se está abordando en este trabajo, estas variables son de tipo ordinal, por lo que se va a hacer uso del clasificador desarrollado en \ref{sec:ord}-\nameref{sec:ord}.
Respecto al pre-procesamiento, se va a usar el mismo proceso explicado en \ref{sec:pre}-\nameref{sec:pre}.
