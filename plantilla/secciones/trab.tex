\chapter{Trabajos Futuros}
El mundo de la ciencia de datos está en constante evolución, surgiendo de nuevas investigaciones modelos nuevos y diferentes enfoques para resolver todo tipo de problemas relacionados con la extracción de conocimiento de un conjunto de datos.\\
\linebreak
Como se ha comentado en la Sección \ref{sec:con}, hemos logrado afirmar que hay conocimiento en este conjunto. Este paso es muy importante, ya que podrán afrontar futuras investigaciones usando este conjunto con la certeza de que no se va a estar trabajando con un conjunto de datos inválido.\\
\linebreak
Por estos motivos, el problema en el que se enfoca este trabajo puede seguir siendo investigado, y a medida que vaya apareciendo nuevos modelos y enfoques para resolver el problema de\textbf{ clasificación multi-etiqueta ordinal}, estos nuevos enfoques podrán ser testeados usando el conjunto de datos con el que se ha estado trabajando.\\
\linebreak
A continuación se muestra una serie de trabajos futuros que se pueden realizar en el futuro para resolver este problema usando enfoques distintos.
\begin{itemize}
	\item\textbf{Inclusión de más datos recientes y provenientes de otros países:}
	\item \textbf{Análisis de estos datos mediante técnicas que estudian la monotonía entre variables explicativas y predictoras. Aprendizaje multi-etiqueta monotónico:}
	\item \textbf{Análisis de la explicabilidad de modelos de Machine Learning y del nivel de sesgo presente en los datos para detectar y evitar decisiones injustas o no apropiadas:}
	\item \textbf{Diseño de un algoritmo ad-hoc para realizar un modelo más concreto y adaptado a estos datos que incorpore los fundamentos aprendidos en en presente análisis:       }
\end{itemize}
