\chapter{Introducción}
Gracias a la evolución tecnológica que hemos experimentado estos últimos años, y sobre todo a la gigantesca expansión del Internet y a la cantidad de datos generados por dispositivos conectados a la red, ha provocado un enorme interés analizar los datos recogidos.\\
Gracias a esta información es los que nos identifican como posibles compradores de un producto, los que hacen que los coches puedan conducir por si mismos o incluso, ayuden al personal médico detectar una patología a partir de ciertos datos del paciente.\\
\linebreak
Se denomina \textbf{ciencia de datos} a la extracción del conocimiento de una gran cantidad datos mediante el uso del software y de técnicas estadísticas. Este conocimiento puede ser usado para \textbf{predecir} cierta salida en función de los datos, para \textbf{detectar grupos} de datos que tengan características similares, \textbf{explicar} el comportamiento de esos datos, entre otros muchos más casos de uso. \\
\linebreak
En el proyecto que se va a estudiar en las secciones siguientes, trata de responder una serie de cuestiones relacionadas con la \textbf{Intención Emprendedora} de una persona.\\
Se podría definir la Intención Emprendedora de una persona como el estado mental que provoca una atención, experiencia y acción hacia un concepto de negocio (Bird. 1998), asumiendo dicha persona no reacciona de forma automática ante los estímulos del medio, sino que procesa la información que le rodea.
\clearpage
\section{Motivación}
En la economía moderna, las empresas juegan un papel muy importante, ya que son una fuente de empleo y riqueza en las economías modernas. Debido al impacto que pueden tener sobre la economía, se debe de apostar por aquellas personas dispuestas a crear una empresa, asegurando que puedan acceder a los recursos formativos y asesoramiento necesario, ya que puede ser un factor muy importante en el proceso de recuperación económica tras la crisis derivada de la pandemia que ha ocurrido estos últimos años.\\
\linebreak
Con esta idea en mente, el trabajo que se ha realizado en este proyecto busca el detectar si una persona tiene una actitud emprendedora, con el objetivo de detectar a aquellas personas más idóneas y que puedan aprovechar al máximo la información a la que pueden acceder. También busca igualar las posibilidades para aquellas personas que tengan esa actitud, pero es posible que tengan dudas a la hora de emprender, pudiendo asesorar correctamente a estas personas.\\
\linebreak
Además de predecir cual es la intención emprendedora de una persona, es muy importante saber cuales son los factores más influyentes a la hora de detectar este tipo de actitudes en las personas, ya que así se puede trabajar en aquellos factores más relevantes, con el objetivo de incrementar el número de personas dispuestas a emprender, ya que según los datos mostrados en \cite{ie}, menos de un $4\%$ de los recién titulados optan por la creación de empresas en su primer empleo. Esta información podría ser usada para que estos recién graduados y graduadas tengan más fácil el camino del emprendimiento y el de la creación de nuevas empresas.\\
\linebreak
El hacer uso de las técnicas de aprendizaje automático junto con los objetivos que se han mencionado previamente, son factores que han influido a la hora de escoger este trabajo.\\
Desde que descubrí el mundo de la inteligencia artificial y la gran cantidad de problemas que usando otro tipos de técnicas no pueden ser resueltos o serían muy costosos en tiempo y recursos, mi interés por conocer este campo y el como funciona empezó a crecer.\\
\linebreak
Todos estos motivos y objetivos, tanto personales como relacionados con el problema, han sido las principales motivaciones de resolver e intentar explicar cuales son los factores más influyentes de este problema.
\clearpage
\section{Objetivos}
\label{sec:obj}
A continuación, se enumeran los distintos objetivos que se han puesto a la hora de realizar el trabajo. \\
\linebreak
El primer objetivo que se plantea es el de comprobar si hay conocimiento en el conjunto de datos que se nos ha proporcionado. Probablemente, este sea uno de los pasos más importante a la hora de afrontar este tipo de problemas, ya que puede ocurrir que se hayan cometido errores a la hora de recoger los datos, no se haya recogido una muestra suficientemente grande para resolver el problema o simplemente, no se pueda abordar con estas técnicas. Ejecutar correctamente este paso es clave, ya que detectando estos problemas (entre otros que pueden ocurrir) antes de seguir profundizando, puede evitar invertir tiempo y esfuerzo.\\
\linebreak
Siguiendo con la lista de objetivos de este trabajo era el de comprobar que preguntas (variables del conjunto de entrenamiento) son relevantes y cuales no. Las variables usadas en todo el proceso de KDD han sido recogidas mediante una encuesta. Estas encuestas pueden resultar tediosas cuando se añaden preguntas que son muy poco relevantes, por lo que la posibilidad de re-plantear ciertas preguntas que se han demostrado que no influyen o que los algoritmos no han encontrado puede ser una buena idea, siempre y cuando se analice el porqué esa pregunta es mala, ya que hay que tener en cuenta la posibilidad de que para un modelo una o varias preguntas no sea relevantes, pero en otro  sí que lo sean.\\
\linebreak
Otro punto de mejora sobre la redacción de estas encuestas es el detectar si un cierto campo ha tenido mucha relevancia, podría ser buena idea plantear la posibilidad de, en un futuro, añadir nuevas preguntas relacionadas con ese campo para comprobar si se puede mejorar el rendimiento de los modelos, teniendo en cuenta (obviamente) que se va a trabajar con conjunto de datos distinto, por lo que puede haber ciertas diferencias.\\
\linebreak
El ultimo objetivo que se ha planteado es el de ser capaces de predecir si una persona tiene una actitud emprendedora o no, obteniendo una información muy importante a la hora de mejorar la selección de posibles en una campaña de marketing especializando las ofertas según la intención emprendedora. 
\clearpage
\section{Asignaturas carrera}
Podría haber buscado un trabajo más fácil, usando unos datos que ya han sido probados con algoritmos de aprendizaje automático y tener la certeza que de una forma u otra, iba a obtener unos resultados buenos siguiendo el mismo proceso que han seguido otras personas que se han enfrentado al problema previamente. Cuando se me propuso este trabajo, en el que podía enfrentarme a un problema nuevo, al que nunca me había enfrentado durante la época de formación en la universidad, la posibilidad de enfrentarme de una manera más "real" a un problema de este tipo, y sobre todo, la oportunidad de aprender nuevos conceptos y técnicas de este campo, fueron razones suficientes por las que decidí realizar este trabajo.\\
\linebreak
Este trabajo se ha realizado desde el principio con la idea en mente de crear un paquete software que agrupe utilidades para que en un futuro, cuando se afronte un problema distinto, tener una buena base general de la que partir, comprobando con  este problema de manera práctica en este trabajo el funcionamiento. Esto no va a quitar el hacer uso de ciertas utilidades que ya han sido probadas por una gran cantidad de personas expertas en el campo, porque uno de los objetivos principales que siempre hay que tener en cuenta a la hora de realizar un desarrollo software es el de hacer uso de los recursos existentes, que como mencionaba antes, han sido probados por la comunidad de desarrolladores.\\ 
\linebreak
Finalmente, la posibilidad de hacer un buen uso de los conocimientos obtenidos tras el continuo esfuerzo que ha sido el de afrontar el grado, han sido las principales motivaciones que me han conducido a afrontar este trabajo.
\clearpage
\section{Estructura de la memoria}
Esta memoria está organizada en distintos capítulos que irán guiando en el desarrollo del trabajo, enfoques usados y cualquier concepto que este relacionado.\\
\linebreak
Se va a comenzar explicando que es el \textit{Machine Learning} y de manera general, conceptos clave que se necesitan tener en cuenta para la comprensión del trabajo. Este capítulo aborda los distintos problemas que se pueden encontrar y como se agrupan los mismos.\\
\linebreak
Después de la explicación de estos conceptos básicos, se va a explicar el problema que se ha abordado en este trabajo, explicando de una forma más específica que tipo de problema es, de donde viene y que elementos lo forman.\\
\linebreak
Explicado los conceptos básicos del problema, se comienza por realizar un análisis de los datos que se han proporcionado. En este sección se explicaran como están formados los datos, y que procesamiento se ha necesitado hacer para que los algoritmos funcionen.
\linebreak
Una vez explicado el problema y el procesamiento previo que se ha realizado, se explican que algoritmos se han usado y cual es la metodología seguida para asegurar unos resultados correctos, evitando caer en las equivocaciones más comunes a la hora de enfrentarse a un problema de este tipo. Esta sección también explicará las distintas formas en las que se han usado para medir el rendimiento de los algoritmos.\\
\linebreak
A continuación, se encuentra un bloque compuesto por varias secciones donde se muestran resultados, análisis de los mismos y variaciones que se han probado y los resultados de las distintas opciones que se han barajado.\\
\linebreak
Mostrado ya los resultados, se continua explicando cual ha sido el proceso software que se ha seguido durante el desarrollo. En esta sección se explicará lenguajes de programación, pautas que se han seguido, herramientas usadas y el proceso de instalación y ejecución del proyecto.\\
\linebreak
Finalmente, se mostrara el bloque de conclusiones y trabajos futuros. Estas secciones finalizaran el proyecto respondiendo a las distintas preguntas que se han hecho durante el desarrollo y valorando si se han cumplido los objetivos propuestos en un inicio.


