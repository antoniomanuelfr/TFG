\section{Introducción}
Gracias a la evolución tecnológica que hemos experimentado estos últimos años, sobre todo a la gigantesca expansión del Internet y  a la enorme cantidad de datos que los dispositivos que están conectados son capaces de  generar, ha provocado un enorme interés en extraer información de  datos. Estos datos son los que nos identifican como posibles compradores de un producto,  los que hacen que los coches puedan conducir por si mismos o incluso, ayuden al personal médico detectar una patología a partir de ciertos datos del paciente.
 \linebreak

Se denomina \textbf{ciencia de datos} a la extracción del conocimiento de una gran cantidad datos mediante el uso del software y de  técnicas estadísticas. Este conocimiento puede ser usado para \textbf{predecir} cierta salida en función de los datos, para \textbf{detectar grupos} de datos que tengan características similares, \textbf{explicar} el comportamiento de esos datos, entre otros muchos más casos de uso. \\
\linebreak
El \textbf{machine learning} es un campo dentro de la ciencia de datos que se centra en la creación y entrenamiento de modelos de predicción con el objetivo de minimizar el error de salida ante una nueva entrada que el modelo no ha usado en la fase de entrenamiento.

\subsection{Introducción al problema}
Pongámonos en situación de una persona dentro del departamento de marketing de una empresa que se centra en la creación de cursos. Determinar que variables son capaces de definir la intención emprendedora de una sección de la población podría ser beneficioso para la empresa, ya que podrían crear esos cursos enfocándose en aquellas variables más importantes (edad, estudios, ......) para que de una forma más eficiente captar a interesados, cambiar ciertas características de esos cursos para que sean accesibles a gente con más interés y, en definitiva, optimizar los recursos empleados en esos cursos para incrementar el beneficio.\\

