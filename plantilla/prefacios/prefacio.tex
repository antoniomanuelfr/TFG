\thispagestyle{empty}

\begin{center}
{\large\bfseries Análisis de la intención de emprendimiento a través de una recogida de datos por encuestas mediante técnicas de preprocesamiento de datos y aprendizaje automático.}\\
\end{center}
\begin{center}
Antonio Manuel Fresneda Rodríguez.\\
\end{center}

%\vspace{0.7cm}

\vspace{0.5cm}
\noindent{\textbf{Palabras clave}: \textit{intención emprendedora}, \textit{aprendizaje automático}, \textit{clasificación ordinal}, \textit{clasificación multi-etiqueta}, \textit{regresión}, \textit{Python}
\vspace{0.7cm}

\noindent{\textbf{Resumen}\\
	\linebreak
	Para que un país sea sólido económicamente, necesita empresas productivas, ya que son las que crean puestos de trabajo e impulsan el desarrollo económico.\\
	Por este motivo, detectar a aquellas personas que son capaces de crear nuevas empresa y promover que estas sean capaces de emprender y facilitar el acceso a recursos y asesoramiento necesarios puede ser un factor importante de cara al crecimiento económico de un país.\\
	La intención emprendedora se puede definir como el estado mental que provoca una atención, experiencia y acción hacia un concepto de negocio, asumiendo que esa persona no reacciona de forma automática antes los estímulos del medio, sino que procesa la información que le rodea (Bird. 1998).\\
	\linebreak
	Resolver este tipo de problemas usando técnicas clásicas puede no ser la mejor opción, ya que puede llegar a ser muy costoso en tiempo y recursos. Con el objetivo de facilitar este tipo de problemas nació el Aprendizaje Automático.\\
	El Aprendizaje Automático hace uso de unos \textbf{datos} y construye un \textbf{modelo} capaz de realizar \textbf{predicciones} tanto de los datos que ha usado como de datos nuevos.\\
	Este campo ha sufrido una gran evolución en estos últimos años, donde se han propuesto nuevos problemas, modelos y metodología, aumentando la cantidad de problemas que pueden ser resueltos.\\
	\linebreak
	Este trabajo hace uso de técnicas de Aprendizaje Automático y de unos \textbf{datos} recogidos sobre un conjunto de la población española y ecuatoriana para intentar predecir si una persona va a tener una intención emprendedora y cuales son los factores más importantes para que una persona tome este tipo de decisiones.
\cleardoublepage

\begin{center}
	{\large\bfseries Analysis of entrepreneurial intention through data collected by surveys using data preprocessing techniques and Machine Learning }\\
\end{center}
\begin{center}
	Antonio Manuel Fresneda Rodríguez\\
\end{center}
\vspace{0.5cm}
\noindent{\textbf{Keywords}: \textit{entrepreneurial intention}, \textit{machine learning}, \textit{ordinal classification}, \textit{multi-label clasification}, \textit{regression}, \textit{Python}
\vspace{0.7cm}

\noindent{\textbf{Abstract}\\
\linebreak
To keep a country solid economically, it will need profitable companies, as these are in charge of creating job posts and pushing the economic growth.\\
For these reasons, detecting people capable of creating new companies and helping them to get the resources and advice, can be important for the economic growth of a country.
Entrepreneurial intention is the mental state of mind that precedes action and directs attention toward entrepreneurial behaviors such as starting a new business. This person will use all the information around him/her and will process it to get the best decision, instead of reacting automatically to the stimuli (Bird 1998).\\
\linebreak
Solving this kind of problem using classic techniques may not be the best choice, as this can be time and resource-intensive. The goal of  Machine Learning is to ease this kind of problem. This approach will use data to build a mathematical model that will predict the data used to create the model and, most important, new data that the model hasn't used.
This field has evolved significantly during the last years, proposing new problems, models, and methodology.\\
\linebreak
This work will use Machine Learning techniques and data collected from Spanish and Ecuadorian people to find the entrepreneurial intention of a person and which are the key factors that make that choice.
\cleardoublepage
\chapter*{}
\thispagestyle{empty}

\noindent\rule[-1ex]{\textwidth}{2pt}\\[4.5ex]

Yo, \textbf{Antonio Manuel Fresneda Rodríguez}, alumno de la titulación Ingeniería Informática de la \textbf{Escuela Técnica Superior de Ingenierías Informática y de Telecomunicación de la Universidad de Granada}, autorizo la ubicación de la siguiente copia de mi Trabajo Fin de Grado en la biblioteca del centro para que pueda ser consultada por las personas que lo deseen.
\vspace{6cm}

\noindent Fdo: Antonio Manuel Fresneda Rodríguez
\vspace{2cm}
\begin{flushright}
	Granada a 17 de Noviembre de 2021 .
\end{flushright}

\cleardoublepage
\thispagestyle{empty}

\noindent\rule[-1ex]{\textwidth}{2pt}\\[4.5ex]

D. \textbf{Salvador García López}, Profesor del Área de Computación y Sistemas Inteligentes del Departamento de Ciencias de la Computación e Inteligencia Artificial de la Universidad de Granada.

\vspace{0.5cm}

\textbf{Informo:}

\vspace{0.5cm}

Que el presente trabajo, titulado \textit{\textbf{ Análisis de la intención de emprendimiento a través de una recogida de datos por encuestas mediante técnicas de preprocesamiento de datos y aprendizaje automático}},
ha sido realizado bajo mi supervisión por \textbf{Antonio Manuel Fresneda Rodríguez}, y autorizo la defensa de dicho trabajo ante el tribunal
que corresponda.

\vspace{0.5cm}

Y para que conste, expiden y firman el presente informe en Granada a Noviembre de 2021.

\vspace{1cm}

\textbf{El/la director(a)/es: }

\vspace{5cm}

\noindent \textbf{Salvador García López}
\pagebreak
\section*{Agradecimientos}
Me gustaría comenzar esta sección agradeciendo a mis padres el esfuerzo y la oportunidad que ellos no tuvieron para disfrutar de una educación universitaria. Muchas gracias por todo y os estoy realmente agradecido. \\
A mis hermanas, que son muy importantes para mi y que siempre han estado ahí para apoyarme y de vez en cuando para cabrearme también.
A toda mi familia, gracias abuela, abuelo, tíos, tías, primos, primas os quiero mucho y sois el pilar más importante de mi vida.\\
\linebreak
No me quiero olvidar tampoco de aquellos profesores que han formado parte en mi educación, desde la secundaria, donde creo que en ese momento no llegué a valorar lo importante que era la formación en ingles que estuve recibiendo, hasta todos aquellos docentes que me han dado clase durante todos estos años en la ETSIIT aprendiendo todo lo que me ha llevado a conseguir un trabajo en estas épocas tan complicadas y que me está haciendo muy feliz. Muchas gracias.\\
Una mención aparte para Salva, que me ha propuesto un trabajo muy interesante, me ha ayudado y animado a realizar este trabajo al que le he dedicado un montón de horas en estos meses. Muchas gracias.\\
\linebreak
Agradecer también a todos mis amigos y amigas que he estado haciendo durante todos estos años, muchas gracias por estar siempre ahí y por ser capaces de hacerme desconectar de vez en cuando.
